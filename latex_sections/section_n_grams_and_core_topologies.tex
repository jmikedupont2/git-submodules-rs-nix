\section{N-grams and Core Topologies}
For each prime $p_i \in \mathcal{P}$, we define a corresponding set of $p_i$-grams. An $n$-gram is a contiguous sequence of $n$ items from a given sample. In this framework, these $p_i$-grams represent our \textit{core topologies}.
For example:
\begin{itemize}
    
    \item For $p_1 = 2$, we consider 2-grams (pairs).
    
    \item For $p_2 = 3$, we consider 3-grams (triples).
    
    \item \ldots
    
    \item For $p_8 = 19$, we consider 19-grams (19-tuples).
\end{itemize}
These n-grams are formed from a source of data, which can be conceptualized as a sequence of elements derived from the properties of our instances.
