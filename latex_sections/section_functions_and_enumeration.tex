\section{Functions and Enumeration}
We consider a set of $M$ distinct functions, $\mathcal{F} = \{f_1, f_2, \ldots, f_M\}$, that can be applied to these instances or their properties. These functions can enumerate various characteristics. For example, a function might count the number of instances that possess a certain bit size:
\[ \Sigma(N_{bits}) = \text{count of instances with bit size } N_{bits} \]
The application of these functions allows for the sampling and analysis of the instances based on metrics such as byte size, number of files, inodes, cache lines, data types, functions, and time spent. The "top N" selection refers to identifying instances or topologies based on these enumerated metrics.
